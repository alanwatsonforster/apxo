\section*{F-80/T-33 Shooting Star}

The Lockheed F-80 Shooting Star was a day fighter and fighter-bomber. It had unswept wings and a single Allison J33 centrifugal-flow jet engine. It entered service with the USAAF shortly before the end of WWII, but did not see combat until the Korean War, in which it served extensively with the USAF as a fighter and fighter bomber. Like other early jet fighters with unswept wings, it was found to have inferior performance to the MiG-15bis and was replaced by variants of the F-86 Sabre. It later served with the air forces of Brazil, Chile, Colombia, Ecuador, Peru, and Uruguay.

The P-80A was the first version to enter service and was armed with six .50~cal M2 machine guns in the nose. It was followed by the P-80B, which replaced the M2 machine guns with faster-firing M3 machine guns, used an ejection seat, and many minor improvements. The P-80C was produced in larger numbers than both previous versions. In USAF service from 1948, the P-80 was designated the F-80.

Like many early jet fighters, the F-80 suffered from short range. This was mitigated by the provision of jettisonable wing-tip fuel tanks. These stations were designed for 165 gal tanks, but these were found to be insufficient for missions over Korea from the USAF bases in Japan. A local field modification produced the 265 gal “Misawa” tank, which gave usefully greater range and loiter time, at the cost of overloading the wing-tip stations.

The F-80C was armed with six .50~cal M3 machine guns with 300 rounds per ground. A typical weapon load for F-80Cs engaged in ground-attack missions in the Korean War was two 500~lb M64 bombs or 110 gal (750~lb) napalm bombs often supplemented by four or eight HVAR rockets on the inner-wing stations. Although bombs could in theory be carried on the wing-tip stations, this does not appear to have occurred in practice. When flying from Japan, the weapon load was often limited to four HVARs rockets.

The RF-80C was an unarmed photo-reconnaissance version of the F-80C, with cameras replacing the machine guns in the nose. It was used by the USAF.

The T-33 Shooting Star (known informally as the “T-Bird”) was a two-seater trainer developed from the F-80. Most T-33As were unarmed, but a number had two .50~cal M3 machine guns. The T-33A was used by the USAF, USN, RCAF, and the air forces of Bangladesh, Belgium, Bolivia, Brazil, Burma, Canada, Chile, Republic of China (Taiwan), Colombia, Cuba, Denmark, Dominican Republic, Ethiopia, Ecuador, El Salvador, France, Federal Republic of Germany, Greece, Guatemala, Honduras, Indonesia, Iran, Italy, Japan, Libya, Mexico, Netherlands, Nicaragua, Norway, Pakistan, Paraguay, Peru, Philippines, Portugal, Saudi Arabia, Singapore, South Korea, Spain, Thailand, Turkey, Uruguay, and Yugoslavia. Cuban T-33As saw combat during the Bay of Pigs invasion.

The RT-33A was a photo-reconnaissance version of the T-33A developed for foreign use. The nose was replaced with one with oblique and vertical cameras and the rear cockpit was used for equipment relocated from the nose of the T-33A and for additional fuel. It was used by the air forces of Belgium, Chile, Colombia, Ethiopia, France, Greece, Italy, Iran, Netherlands, Pakistan, Portugal, Saudi Arabia, Taiwan, Turkey, Thailand, and Yugoslavia. One was also used by the USAF for Project Field Goal, clandestine reconnaissance missions over Laos in the early 1960s.

Some T-33As were converted into AT-33A light attack aircraft by adding under-wing pylons to the machine guns. The AT-33A was used as a trainer by the USAF and also by the air forces of Brazil, Burma, Dominican Republic, Ecuador, Greece, Mexico, Nicaragua, Paraguay, and Uruguay.

ADCs are provided for the:
\begin{itemize}
\item F-80C
\item RF-80C
\item T-33A
\item RT-33A
\item AT-33A
\end{itemize}
