\section*{Sea Fury}

The Hawker Sea Fury was a post-WW2 propeller-driven fighter bomber. 

The Sea Fury was a descendant of the WW2 Hawker Tempest fighter bomber, originally adapted as a long-range fighter for use in the war against Japan. After WW2 ended, the RAF no longer had interest, but the RN acquired a version adapted for carrier operations as a replacement for its Seafires, which were not well suited to carrier operations, and Corsairs, which had to be returned to the US as the Lend-Lease program ended. At the time, the high landing speeds of the first generation of jet aircraft was an impediment to their use on carriers. The Sea Fury was powered by a Centaur radial engine, armed with four 20 mm guns, and had a bubble canopy with an excellent view except under the nose.

There were three major versions of the Sea Fury. The first was the F.10 day fighter, which entered service with the RN in 1947. This was quickly followed by the FB.11, which added armor and weapon stations to fulfill the fighter-bomber role better. In 1950, a two-seater T.20 trainer was deployed. The Sea Fury F.50, FB.50, and FB.51 and the Fury FB.60 were export versions of the F.10 and FB.11 with minor changes and removal of carrier-specific equipment from the FB.60. 

The Sea Fury was exported to Australia, Burma, Canada, Cuba, Iraq, the Netherlands, and Pakistan. The RN and RAN used it as a fighter-bomber in the Korean War and also saw combat with the Cuban Revolutionary Air Force during the Bay of Pigs invasion and with the Netherlands Royal Navy in the Dutch East Indies. It was replaced in the RN by the Sea Hawk and Attacker starting in 1953, in the RCN by the F2H Banshees starting in 1956, and in the Netherlands Royal Navy by Sea Hawks.

A typical air-to-ground load would be two 500- or 1000-lb bombs or twelve RP-3 rockets. It could also carry two 90-gallon fuel tanks to extend its range. Alternatively, it could be equipped with cameras for photographic reconnaissance missions.

An ADC is provided for the:
\begin{itemize}
\item Sea Fury FB.10
\end{itemize}
