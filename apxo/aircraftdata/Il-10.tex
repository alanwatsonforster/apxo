\section*{Ilyushin Il-10}

The Ilyushin Il-10 is a propeller-driven ground-attack aircraft developed during WWII as a replacement for the famous Il-2 Sturmovik. It entered service with the Soviet Union in 1944. After the war, it was supplied to many Soviet allies and was also manufactured under license in Czechoslovakia.

Like its predecessor, the Il-10 features extensive armor for the crew and engine and a ventral defensive gun to protect from attacks from the rear. The initial Il-10 has two 23 mm VYa-23 cannons and two 7.62 mm ShKAS machine guns in the wings, with 150 and 750 rounds each, respectively, plus a 12.7 mm UBT machine gun with 150 rounds protecting the tail. Later versions have four 23 mm NR-23 cannons in the wings and a 20 mm B-20 cannon in the ventral mount.

A typical weapon load would be four 100 kg bombs, two in the bomb bays and two on the wing stations, or two 250 kg bombs on the wings. Alternatively, the wing stations could carry a total of eight RS-82 or four RS-132 rockets (although apparently these rockets were not used in the Korean War).

The Il-10 was used by the North Korean KPAF and saw extensive service in the first weeks of the Korean War, before suffering large losses against USAF fighters. They were also used by the Chinese PLAAF.

ADCs are provided for:
\begin{itemize}
\item Il-10
\end{itemize}
