\section*{Avon Sabre}

The CAC Avon Sabre is a day fighter and fighter-bomber derived from the North American F-86F Sabre with a more powerful Rolls-Royce Avon 20 engine and two 30 mm ADEN guns. It has a secondary role as a fighter-bomber. The Mk.32 version competes with the Canadair Sabre 6 for the honor of being the very best day-fighter Sabre.

The Mk.30 had the original slatted wing from the A, E, and early F versions of the F-86. It was introduced into the RAAF in 1954, replacing the Meteor F.8.

The Mk.31 was introduced in 1955. The significant improvement was the adoption of the unslatted 6-3 wing from later F versions of the F-86, which gave better performance at higher speeds and altitudes. All surviving Mk.30 were subsequently upgraded to the Mk.31 level. 

Finally, the Mk.32 was introduced in 1956. It gained an Avon 26 engine, with modifications to prevent surges when the guns were fired, and two weapon stations that allowed it to carry air-to-ground weapons and fuel tanks simultaneously, giving it a longer range when employed as a fighter-bomber.

In 1960, the RAAF adopted the AIM-9B IRM for both the Mk.31 and Mk.32. The Mk.31 also received two additional weapon stations to allow it to carry missiles or bombs and fuel tanks simultaneously, like the Mk.32.

Avon Sabres served with the RAAF in the Malayan emergency, the Malaysian-Indonesian confrontation, and the Vietnam War in Thailand. The last Avon Sabres left RAAF service in 1971, having been replaced by Mirage IIIs, but a number subsequently served with the Malaysian and Indonesian air forces.

A typical air-to-air load would be two 167 gal (750L) fuel tanks on the outer stations and, from 1960, two AIM-9Bs on the inner stations. For air-to-ground, two 1000 lb bombs might be carried on the inner stations along with two 167 gal tanks on the outer stations. Alternatively, the Mk.32 could carry twenty-four Hispano Sura 80R rockets, with three carried one below the other on each of the eight rocket stations, but at the price of not being able to use external fuel tanks. For ferry flights, two 100 gal (400L) fuel tanks could be carried on the inner stations and two 167 gal (750L) fuel tanks to the outer ones.

ADCs are provided for:
\begin{itemize}
\item Avon Sabre Mk.30
\item Avon Sabre Mk.31
\item Avon Sabre Mk.31 (Late) –– 1960 update for AIM-9B IRMs
\item Avon Sabre Mk.32
\end{itemize}
