\section*{B-29 and RB-29 Superfortress}

The Boeing B-29 Superfortress was a strategic bomber. It entered service in the USAAF before the end of WWII and saw combat in the Pacific Theaters. It later served with the USAF in the Korean War.

The defensive armament of the B-29 was four remote-control turrets, two dorsal and two ventral, each equipped with two .50~cal M2 machine guns and a tail station with two .50~cal M2 machine guns and one 20 mm M2 cannon. Later models had four .50~cals in the forward dorsal turret. The ballistic characteristics of the 20 mm were not well-matched to the .50~cals, and it was later removed or replaced by a third .50~cal. Each gun has 500 rounds of ammunition. 

The RB-29A is a strategic photo-reconnaissance version of the B-29A. It retained full combat capability.

The Silverplate and Saddletree versions were adapted for delivery of nuclear bombs.  The Silverplate program began during WWII --- Silverplate aircraft dropped nuclear bombs on Hiroshima and Nagasaki --- but was superseded by the similar Saddletree program in 1947. These aircraft were converted by removing the four turrets and their associated fire-control system and all armor removed and installing equipment for nuclear weapons and additional fuel tanks. The Saddletree aircraft also were equipped for air-to-air refueling.

A typical bomb load for the conventional variant during the Korean War was twenty 500~lb M64 or 1,000~lb M65 bombs. Occasionally, 2,0000~lb M66 and 4,000~lb M56 bombs were used. 

The Silverplate and Saddletree variants could carry a single Mark 3, 4, or 6 nuclear bomb.

ADCs are provided for:
\begin{itemize}
\item B-29A
\item RB-29A
\item B-29A (Silverplate)
\item B-29A (Saddletree)
\end{itemize}
