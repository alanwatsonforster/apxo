\section*{Avro Canada CF-100 Canuck}

The Avro Canada CF-100 is a twin-engined, straight-wing, all-weather interceptor designed specifically for the RCAF to intercept Soviet bombers at long range over Canada. It is similar in many respects to the F-89 Scorpion and F-94 Starfire, but is larger than both.

The first production version was the Mk 3, which entered service in 1953. It was equipped with the APG-33 radar and Hughes E-1 fire-control system (which were also used on the F-94A/B), and armed with eight .50 cal M3 machine guns in a ventral pack.

The Mk 4A used more powerful Orenda 9 engines. The radar and fire-control system were upgraded to the APG-40 and Hughes MG-2 (which were also used on the F-89D). The main armament was wing-tip pods each with 29 FFAR rockets, although it retained the ventral gun pack. The pods could be swapped with fuel tanks for ferry flights. The Mk 4A entered service in 1953. The Mk 4B was similar, but used more powerful Orenda 11 engines.

The Mk 5 had extended wings and horizontal stabilizers for better performance at high altitude. The gun pack was no longer considered to be effective for attacking bombers and so was omitted to save weight. It entered service in 1955.

The CF-100 served with the RCAF in Canada from 1953 to 1962, and began to be replaced by the CF-101 from 1961. They also served with the RCAF in Europe from 1956 to 1962. The Belgian Air Force also flew Mk 5s from 1957 to the early 1960s.

ADCs are provided for:
\begin{itemize}
\item CF-100 Mk 4B
\item CF-100 Mk 5
\end{itemize}
