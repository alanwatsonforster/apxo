\chapter*{Mikoyan-Gurevich MiG-15}

The Mikoyan-Gurevich MiG-15 is an early jet interceptor and day fighter. The MiG-15 was one of the first fighters to incorporate a swept wing and a swept tail to reduce the effects of compressibility and allow higher transonic speeds. However, it used conventional elevators and so became difficult to control at high speeds. The NATO reporting name is “Fagot.”

When used as a fighter, it was limited by the slow rate of fire of these guns (400 and 800 rounds per minute, respectively), but a single hit on a fighter or fighter-bomber could often inflict fatal damage.

The initial MiG-15 was powered by the Klimov RD-45, an unauthorized copy of the Rolls-Royce Nene engine (which had been sold to the Soviet Union in small numbers with the agreement of the British Government). As it was initially intended to serve as an interceptor, it carried a very gun armament of one 37 mm N-37 cannon with 40 rounds and two 23 mm NS-23 cannon with 80 rounds per gun, both under the nose. The NATO reporting name is “Fagot-A.”

The MiG-15bis was improved in a number of small but important ways. The engine was upgraded to the Klimov VK-1, a development of the RD45 with more power. The NS-23 cannons were replaced by faster-firing NR-23 cannons. The NATO reporting name is “Fagot-B.”

The MiG-15P was a prototype all-weather interceptor, equipped with the RP-1 Izumrud radar (NATO reporting name “Scan Fix”) and with its armament reduced to two 23 mm NR-23 cannons.

The MiG-15ISh is a prototype attack version with pylons in the wings for bombs or rockets in addition to the stations for fuel tanks.

ADCs are provided for:
\begin{itemize}
    \item MiG-15bis
    \item MiG-15P
    \item Mig-15ISh
\end{itemize}
