\section*{Tupolev Tu-4}

The Tupolev Tu-4 was a strategic bomber. It was largely reverse-engineered from Boeing B-29As that had made emergency landings in the USSR during WWII and were subsequently interned. It was, however, fitted with Soviet Shvetsov ASh-73 engines and auxiliary equipment. Furthermore, the .50 cal machine guns on the original B-29A were replaced in the Tu-4 by more powerful 23 mm Nudelman NS-23 cannons, with two in each turret and two in the tail position. The ammunition load is estimated to be 200 rounds per gun. The NATO reporting name for the Tu-4 is Bull.

The Tu-4 entered service with the VVS in large numbers in 1949. As it provided the Soviet Union for the first time with the ability to conduct a one-way strike on peripheral cities in the continental US, including Los Angeles and Chicago, it spurred the development and deployment of interceptors in the United States. This effort gained urgency in 1949 when the Soviet Union demonstrated its first nuclear bomb.

The Tu-4 version was a conventional bomber.

The Tu-4A version was a nuclear bomber. It could carry the RDS-1, -3, and -5 nuclear bombs.

ADCs are provided for
\begin{itemize}
    \item Tu-4
    \item Tu-4A
\end{itemize}