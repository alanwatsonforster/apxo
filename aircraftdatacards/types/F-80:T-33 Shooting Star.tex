\section*{F-80/T-33 Shooting Star}

The Lockheed F-80 Shooting Star is a day fighter. It has unswept wings and a single Allison J33 centrifugal-flow jet engine. It entered service with the USAAF shortly before the end of WWII, but did not see combat. It served extensively with the USAF in the Korean War, as a fighter and fighter bomber; it had inferior performance to the MiG-15bias and was replaced in both roles by variants of the F-86 Sabre. It later served with the air forces of Brazil, Chile, Colombia, Ecuador, Peru, and Uruguay.

The P-80 was armed with six .50 cal M3 machine guns in the nose, but did not have a radar-ranging gunsight. The P-80A was the first version to enter service. It was followed by the P-80B, which had an ejection seat and many minor improvements. The P-80C was produced in larger numbers than both previous versions. In USAF service 1948, the P-80 was designated the F-80.

Like many early jet fighters, the F-80 suffered from short range. This was mitigated by the provision of jettisonable wing-tip fuel tanks. These stations were designed for 165 gal tanks, but these were found to be insufficient for missions over Korea from the USAF bases in Japan. A local field modification produced the 265 gal “Misawa” tank, which gave usefully greater range and loiter time, at the cost of overloading the wing-tip stations.

A typical weapon load for F-80Cs engaged in ground-attack missions in the Korean War would be two 500~lb M64 bombs or eight HVAR rockets on the underwing stations. Although bombs could in theory be carried on the wing-tip stations, this does not appear to have occurred in practice.

The RF-80C was an unarmed photo-reconnaissance version of the F-80C, with cameras replacing the machine guns in the nose.

The T-33 Shooting Star (also known as the “T-Bird”) was a two-seater trainer developed from the F-80. Most T-33As was unarmed, but a number had two .50 cal M3 machine guns. The T-33A was used by the USAF, USN, RCAF, and the air forces of Bolivia, Cuba, Brazil, Greece, Japan, Turkey, and Thailand, among others. Cuban T-33As saw combat during the Bay of Pigs invasion.

A small number were also AT-33A light attack aircraft, adding under-wing pylons to the machine guns. The AT-33A was used as a trainer by the USAF and also by the air forces of Brazil, Burma, the Dominican Republic, Ecuador, Greece, Mexico, Nicaragua, Paraguay, and Uruguay.

ADCs are provided for the:
\begin{itemize}
\item F-80C
\item RF-80C
\item T-33A
\item AT-33A
\end{itemize}
