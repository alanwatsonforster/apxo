\section*{B-29 and RB-29 Superfortress}

The Boeing B-29 Superfortress is a strategic bomber. It entered service in the USAAF before the end of WWII and saw combat in the Pacific Theaters. It later served with the USAF in the Korean War.

The B-29 was designed with four remote-control turrets, two dorsal and two ventral, and each equipped with two .50 cal M2 machine guns and a tail station with two .50 cal M2 machine guns and one 20 mm M2 cannon. The 20 mm was later removed or replaced by a third .50 cal, as its ballistic characteristics were not well-matched to the .50 cal. The turrets were removed the Silverplate and Saddleback variants adapted for the delivery of nuclear weapons. 

The RB-29A is a strategic photo-reconnaissance version of the B-29A. It retained full combat capability.

A typical bomb load for the conventional variant during the Korean War was twenty 500 lb M64 GP bombs. The Silverplate and Saddleback variants could carry a single Mark 3, 4, or 6 nuclear bomb.

\begin{itemize}
\item B-29A
\item B-29A (Silverplate)
\item B-29A (Saddletree)
\item RB-29A
\end{itemize}
