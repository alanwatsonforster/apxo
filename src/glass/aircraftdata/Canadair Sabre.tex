\section*{Canadair Sabre}

The Canadair Sabre is a day fighter derived from the North American F-86 Sabre. The initial versions were only lightly modified, but later versions incorporated the more powerful Orenda engine. The Sabre Mk.6 competes with the Avon Sabre Mk.32  for the honor of being the very best day-fighter Sabre.

The single Sabre Mk.1 was a prototype very similar to the F-86A. 

The first production version was the Mk.2, which was essentially a F-86E. The Mk.4 was very similar. Both of these versions were built with the original Sabre slatted wing.

The Canadair Sabres began to diverge from the North America originals with the Mk.3 prototype, which used a Avro Canada Orenda 3 engine with significantly more thrust than that of the Mk.2. This prototype was subsequently developed into the production Mk.5 version with the improved Orenda 10 engine and the unslatted 6-3 wing. The last version was the Mk.6 with the even more powerful Orenda 14 engine and the slatted 6-3 wing. All of the production versions were fitted with the original F-86 armament of six .50~cal M3 guns.

Sabre Mk.2s were used by the RCAF in Europe and the USAF. Later, they were then passed on to the air forces of Greece and Turkey. In USAF service, they saw combat in the Korean War.

Sabre Mk.4s were used in small numbers by the RCAF and in larger numbers by the RAF, serving alongside the Meteor F.8 and being the first swept-wing fighter in British service. Starting in 1954, they begin to be refitted with the 6-3 wing. As Hawker Hunters became available in 1956, the RAF Sabres were transfered to the Yugoslav and Italian air forces.

Sabre Mk.5s were initially used by the RCAF, again mainly in Europe, replacing the Mk.2s. A number were later transfered to the Luftwaffe.

The Mk.6s in turn replaced the RCAF Mk.5s and also were used in large numbers by the Luftwaffe. These were later sold on to the Columbia, South Africa, and Pakistan. In PAF service they fought in the 1971 war with India.

A typical air-to-air load would be two 200 gal (750L) fuel tanks on the outer stations and, from 1960 on the Mk.6, two AIM-9Bs on the inner stations. For air-to-ground, two 1000 lb bombs might be carried on the inner stations along with two 200 gal tanks on the outer stations. Alternatively, on the later versions, sixteen HVAR rockets could be carried without fuel tanks. For ferry flights, two 120 gal (400L) tanks could be carried on the inner stations and two 200 gal (750L) tanks to the outer ones.

ADCs are provided for the:
\begin{itemize}
\item Sabre Mk.2
\item Sabre Mk.4
\item Sabre Mk.4 (6-3 Wing) --- the Sabre Mk.4 refitted with the 6-3 wing
\item Sabre Mk.5
\item Sabre Mk.6
\end{itemize}
